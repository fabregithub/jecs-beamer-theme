%!TEX TS-program = xelatex
%!TEX encoding = UTF-8 Unicode

\documentclass[aspectratio=169]{beamer}

\usepackage{amssymb} %maths
\usepackage{amsmath} %maths

\usepackage{lipsum}

% to use non-standard font
\usepackage[utf8]{inputenc} %useful to type directly 
\usepackage{zxjatype}
\usepackage{fontspec}
\usepackage[math-style=TeX]{unicode-math}
\setmainfont[Ligatures=TeX,mallCapsFeatures={Letters=SmallCaps}]{Lato}
\setmathfont[math-style=ISO,bold-style=ISO,vargreek-shape=TeX]{Lete Sans Math}
%\setsansfont[Ligatures=TeX]{TeX Gyre Heros}
\setjamainfont{Noto Sans JP}
%\setjasansfont{IPAexGothic}
%\setjamonofont{IPAexGothic}
\usepackage{xltxtra}

%\setmathfont[math-style=ISO,bold-style=ISO]{Lete Sans Math}


\usetheme{JECS}

%\graphicspath{{images/}}

\title{プレゼンテーション}
\subtitle{副題}
\author{著者氏名}
\institute{所属}
\date{\today}

\begin{document}

\titleframe

\begin{frame}[t]{リスト}
子どもの健康と環境に関する全国調査(エコチル調査)は、2011年に始まった全国規模の出生コホート調査です。
	\begin{itemize}
		\item エコチル調査について
			\begin{itemize}
				\item \textbf{参加者リクルート}: 2011年1月--2014年3月
				\item \textbf{エリア}: 日本全国
				\item \textbf{登録参加者数}: 母親103,097人、子ども100, 108人、父親51,909
				\item \textbf{参加機関}: 環境省、国立環境研究所、国立成育医療研究センター、15箇所のユニットセンター
			\end{itemize}
		\item 調査内容
			\begin{itemize}
				\item \textbf{全体調査}: すべての参加者(質問票調査、生体試料収集、医師診察等)
				\item \textbf{詳細調査}: 全体調査からランダムに選ばれた5,000人を対象とした、より詳細な調査
				\item \textbf{追加調査}: 外部資金で実施する調査
			\end{itemize}
		\item Credits: \url{https://www.env.go.jp/chemi/ceh/}
	\end{itemize}
\end{frame}

\begin{frame}[t]{ブロック}
	%\framesubtitle{This is a subtitle}
	\begin{block}{標準ブロック}
		標準ブロックです。
	\end{block}
	
	\begin{exampleblock}{例示ブロック}
		例示のブロックです。
	\end{exampleblock}
	
	\begin{alertblock}{強調ブロック}
		強調時のブロックです。
	\end{alertblock}
\end{frame}

\begin{frame}[t]{数式}
	ベイズ理論は近代疫学の最も便利なツールの一つです。
	\begin{align*}
		P(A | B) &= \frac{P(B | A)\cdot P(A)}{P(B)} \\
		\intertext{ここで}\\
		A,B &= \text{事象}\\
		P(A | B) &= \text{Bが真の時のAの起こる確率}\\
		P(B | A) &= \text{Aが真の時のBの起こる確率}\\
		P(A),P(B) &= \text{互いに独立なAまたはBの起こる確率}
	\end{align*}

\end{frame}

\begin{frame}[t]{二段組み}
	スライドの中で、二段組みもできます。
	\begin{columns}[t]
		\begin{column}[T]{0.4\textwidth}
			\lipsum[1][1-2]
			\vspace{1em}
			\begin{block}{ブロック}
				これはブロックです。
			\end{block}
		\end{column}
		\begin{column}[T]{0.4\textwidth}
			\begin{itemize}
				\item 項目1
				\item 項目2
				\item 項目3
			\end{itemize}
		\end{column}
	\end{columns}
\end{frame}

\begin{frame}[t]{謝辞}
謝辞です。
\end{frame}


\end{document}