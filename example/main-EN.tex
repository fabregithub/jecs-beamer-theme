%!TEX TS-program = xelatex
%!TEX encoding = UTF-8 Unicode

\documentclass[aspectratio=169]{beamer}

\usepackage{amssymb} %maths
\usepackage{amsmath} %maths

\usepackage{lipsum}

% to use non-standard font
\usepackage[utf8]{inputenc} %useful to type directly 
%\usepackage{zxjatype}
\usepackage{fontspec}
\usepackage[math-style=TeX]{unicode-math}

% to ensure that beamer does not meddle with the fonts we use
%\usefonttheme{professionalfonts}

% our main font
\setmainfont[Ligatures=TeX,SmallCapsFeatures={Letters=SmallCaps}]{Lato}
\setmathfont[math-style=ISO,bold-style=ISO,vargreek-shape=TeX]{Lete Sans Math}
%\setmathfont[math-style=ISO,bold-style=ISO]{Lete Sans Math}


\usetheme{JECS}

%\graphicspath{{images/}}

\title{My Presentation}
\subtitle{Subtitle here}
\author{Author Name}
\institute{Affiliation(s)}
\date{\today}

\begin{document}

\titleframe

\begin{frame}[t]{Slide with Lists}
Japan Environment and Children's Study, or JECS, is a nationwide birth cohort study started in 2011.
	\begin{itemize}
		\item JECS in figures
			\begin{itemize}
				\item \textbf{Participant recruitment}: January 2011--March 2014
				\item \textbf{Coverate}: Entire Japan
				\item \textbf{Number of registered participants}: 103,097 mothers, 100, 108 children, 51,909 fathers
				\item \textbf{Organisation}: Ministry of the Environment, National Institute for Environmental Studies, National Centre for Child Health and Development, 15 Regional Centres
			\end{itemize}
		\item Study structure
			\begin{itemize}
				\item \textbf{Main Study}: All the participants (questionnaire administration, biological sample collections, physical examinations)
				\item \textbf{Sub-cohort Study}: 5,000 randomly selected from Main Study
				\item \textbf{Adjunct Study}: conducted with extramural funding
			\end{itemize}
		\item Credits: \url{https://www.env.go.jp/chemi/ceh/en/index.html}
	\end{itemize}
\end{frame}

\begin{frame}[t]{Blocks}
	%\framesubtitle{This is a subtitle}
	\begin{block}{Standard Block}
		This is a standard block.
	\end{block}
	
	\begin{exampleblock}{Example Block}
		This is an example block.
	\end{exampleblock}
	
	\begin{alertblock}{Alert Block}
		This is an alert block.
	\end{alertblock}
\end{frame}

\begin{frame}[t]{Math}
	Bayesian theorem is one of the most useful tools in modern epidemiology.
	\begin{align*}
		P(A | B) &= \frac{P(B | A)\cdot P(A)}{P(B)} \\
		\intertext{where}\\
		A,B &= \text{events}\\
		P(A | B) &= \text{probability of A given B is true}\\
		P(B | A) &= \text{probability of B given A is true}\\
		P(A),P(B) &= \text{independent probabilities of A and B}
	\end{align*}

\end{frame}

\begin{frame}[t]{Two Columns}
	We can also add two columns in the slides.
	\begin{columns}[t]
		\begin{column}[T]{0.4\textwidth}
			\lipsum[1][1-2]
			\vspace{1em}
			\begin{block}{Block}
				I am a block in a column.
			\end{block}
		\end{column}
		\begin{column}[T]{0.4\textwidth}
			\begin{itemize}
				\item In this column,
				\item we just add the
				\item bullet points.
			\end{itemize}
		\end{column}
	\end{columns}
\end{frame}

\begin{frame}[t]{Acknowledgements}
This theme is inspired by Flip theme (creator: Flip Tanedo). The official beamer user guide was also very handy during the development.
\end{frame}


\end{document}